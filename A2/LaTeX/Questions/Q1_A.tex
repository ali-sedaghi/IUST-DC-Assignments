\SubProblem
{فرکانس پیوند فروسو و فراسو در \lr{FDD}}
{
تولید موج و انتقال سیگنال با فرکانس بالا نیازمند توان و برق بیشتر در
\lr{Transmit Amplifier}
است. 
در مخابرات زمینی که هم فرستنده و هم گیرنده روی زمین قرار دارند، توان و برق بیشتری در اختیار آنتن فرستنده است زیرا با یک دکل و آنتن بزرگ در طرف هستیم که متصل به برق شهری است. هر چه توان و برق بیشتری در اختیار باشد می‌توانیم فرکانس بالاتری ایجاد کنیم. پس در مخابرات زمینی فرکانس پیوند فروسو بیشتر از فراسو است. همچنین در سمت کاربر ما با یک تلفن، لپتاپ و ... با توان کم در طرف هستیم زیرا به یک منبع تغذیه محدود مانند باطری متصل است و نمی‌تواند فرکانس بالایی تولید کند، پس فرکانس پیوند فراسو کم می‌باشد.

در مخابرات ماهواره‌ای اما سیگنال‌ها بایستی از جو عبور کنند و فاصله زیادی را طی کنند. این موارد باعث می‌شود پدیده میرایی با شدت بیشتری رخ دهد. سیگنال‌هایی که دارای فرکانس پایین هستند توسط جو منعکس می‌شوند و نمی‌توانند از جو نفوذ کنند و به ماهواره برسند. به همین دلیل بایستی فرکانس در پیوند فراسو بیشتر شود.
بنابراین در مخابرات ماهواره‌ای فرکانس پیوند فراسو می‌تواند بیشتر از فرکانس پیوند فروسو باشد.
در واقع در این نوع مخابرات با یک ماهواره و یک دکل بزرگ روی زمین در طرف هستیم زیرا ایجاد فرکانس بالا در گوشی انرژی زیادی نیاز دارد. و ارتباط میان کاربر و ماهواره با واسطه شکل می‌گیرد.
}